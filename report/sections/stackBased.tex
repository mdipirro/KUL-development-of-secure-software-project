\section{Stack-Based Buffer Overflow}
\subsection{Casper4}
\subsubsection{Description and vulnerability}
Casper4 requires the user to provide a command-line parameter. If this parameter is not provided, it shows an error message and exits with code 1. Otherwise, it calls the \texttt{greetUser} function. \texttt{greetUser} declares a 666-characters long buffer and copies the user-provided parameter in it. Afterwards, it prints \textit{Hello} followed by the buffer's content.

Casper4 has an immediate vulnerability since it uses the unsafe C's \texttt{strcpy} function. It does not check the length of string to be copied to make sure it is not bigger than the buffer's size. This makes a buffer overflow attack possible. One can simply provide a parameter longer than 666 characters and overwrite the buffer (until the return address), potentially injecting code to be executed after the function return. This is possible since there are no stack canaries, which detect a ''naive`` overflow, and the stack is executable. With a non-executable stack this simple version would not be feasible, and a more advanced exploit would be required (see Section~\ref{sec:dataOnly}).

\subsubsection{Exploit}
The exploit for Casper4 is simple. As aforementioned, it works as follows:
\begin{enumerate}
	\item Inserts in the buffer some no-operation (NOP) bytes (\texttt{$\backslash$x90});
	\item Injects the shell-code corresponding to \texttt{/bin/xh} (which length is 21 bytes);
	\item Overwrites the portion of the function's stack frame between the end of the buffer and the return address;
	\item Overwrites the return address making it point to a generic address within the buffer (but prior to the shell-code beginning).
\end{enumerate}

In order to calculate the ``concrete'' addresses, I used \texttt{gdb}. Please note that \texttt{gbd} adds two environment variables, \texttt{LINES} and \texttt{COLUMNS}, which I unset during the debug procedure. When \texttt{greetUser} is executed, the stack situation is as follows:
\begin{itemize}
	\item Buffer beginning address: \texttt{0xbffff0d6}
	\item Local variables beginning address: \texttt{0xbffff378}
	\item Address of the return pointer: \texttt{0xbffff37c}
\end{itemize}
The latter can be obtained by the beginning address of the local variables by subtracting 4 bytes (the function has only one parameter, which is a pointer $\rightarrow$ 4 bytes).

It is now straightforward to determine how many bytes are necessary to overwrite the return address:

\begin{center}
	\texttt{address\_return\_pointer} - \texttt{address\_buffer} + 4 =\\
	= \texttt{0xbffff378} - \texttt{0xbffff0d6} + 4 = 682
\end{center}

The first subtraction determines the number of bytes between the beginning of the address and the address containing the return pointer. The addition takes care of the bytes needed to overwrite the former address. These 682 bytes are filled in as follows:
\begin{itemize}
	\item 645 bytes of NOP (\texttt{$\backslash$x90});
	\item 21 bytes for the shell-code;
	\item 12 random character to reach the target address (''f``);
	\item 4 bytes for a random address within the buffer space, \texttt{0xbfff1d6} in the exploit (256 bytes after the beginning).
\end{itemize}

In order to run the exploit it is enough to run \texttt{make exploit4}.

\subsubsection{Possible countermeasures}
As aforementioned, this vulnerability is made possible thanks to the absence of stack canaries and non-executable stack. With the former, it is easy to detect an overflow, since the only way to overwrite the return address is overflowing the buffer. With the latter, the program exits immediately due to an attempt to execute code taken from a non-executable area. In both cases, the exploit would not work. 

Besides this two simple countermeasures, there are other, more ''expensive``, possibilities. These are control-flow enforcement and data layout randomization. With the former it is easy to detect that the return address is different from the original, statically computed, one. With the latter, it is more difficult to obtain a working exploit, because the stack's layout is randomized. Nonetheless, since the buffer is 666 characters long and the shell-code only occupies 21 bytes, it would still be possible to obtain a working version.

Of course, even a source-code based countermeasure is effective. The vulnerability is based on the fact that, at \texttt{strcpy} time, lengths are not checked. If the source code checked that the length of the parameter was less or equal than 666 before copying, the vulnerability would disappear. More recent string copying C functions perform this kinds of sanity checks.

\subsection{Casper40}\label{sec:casper40}
Since the exploit makes use of the NOP byte to reach the shell-code without compromising the stack, I chose Casper40 as an advanced level. It performs an additional check, prior to the copy, to make sure that the parameter does not contain NOP bytes. 

To counterattack this level of defense, I substitute the sequence of NOP with a pair of assembly instructions, \texttt{$\backslash$x43$\backslash$x4b}, corresponding to \texttt{inc ebx} and \texttt{dec ebx}. They increment, and immediately decrement, the register \texttt{ebx}. Hence, they have no effect neither on \texttt{ebx} nor on the stack. Please note that there is a plenty of pairs eligible to this end. Another possibility would be pushing, and immediately pulling, a register onto the stack. Table~\ref{tbl:casper40} lists some combinations for the registers \texttt{eax}, \texttt{ebx}, \texttt{ecx}, and \texttt{edx}:
\begin{table}[h]
	\centering
	\begin{tabular}{|c|c|c|}
		\hline
		& \texttt{inc dec}                        & \texttt{push pop}                       \\ \hline
		\texttt{eax} & \texttt{$\backslash$x40$\backslash$x48} & \texttt{$\backslash$x50$\backslash$x58} \\ \hline
		\texttt{ebx} & \texttt{$\backslash$x43$\backslash$x4b} & \texttt{$\backslash$x53$\backslash$x5b} \\ \hline
		\texttt{ecx} & \texttt{$\backslash$x41$\backslash$x49} & \texttt{$\backslash$x51$\backslash$x59} \\ \hline
		\texttt{edx} & \texttt{$\backslash$x42$\backslash$x4a} & \texttt{$\backslash$x52$\backslash$x5a} \\ \hline
	\end{tabular}
	\caption{Examples of eligible pairs of instructions for Casper40}
	\label{tbl:casper40}
\end{table}

The memory status when \texttt{greetUser} is executed is as follows:
\begin{itemize}
	\item Buffer beginning address: \texttt{0xbffff0d2}
	\item Local variables beginning address: \texttt{0xbffff378}
	\item Address of the return pointer: \texttt{0xbffff37c}
\end{itemize}
As before, now it is possible to obtain the number of bytes necessary to exploit the program (this time taking care of the additional local variable):

\begin{center}
	\texttt{address\_return\_pointer} - \texttt{address\_buffer} + 4 + 4 =\\
	= \texttt{0xbffff378} - \texttt{0xbffff0d2} + 4 + 4 = 686
\end{center}


The only difference when compared to Casper4 is the second $+ 4$. This takes care of the pointer \texttt{ptr}, placed above the buffer in the stack. These bytes are filled in as follows:
\begin{itemize}
	\item 1 byte for a random character (``f'');
	\item 648 bytes for 324 pairs of instructions without effect (\texttt{$\backslash$x43$\backslash$x4b});
	\item 21 bytes for the shell-code;
	\item 12 random bytes (``f'');
	\item 4 bytes for a random address within the buffer space, \texttt{0xbfff1d6} in the exploit.
\end{itemize}

In order to run the exploit it is enough to run \texttt{make exploit40}.

\subsection{Further levels}
In order to further discuss possible countermeasures, it is worthy to try to exploit Casper41 and Casper42, to prove that ad-hoc countermeasures do not work very well. The code required to exploit both of them is exactly the same as Casper4. They apply ad-hoc countermeasures to prevent specific attacks, but are limited to those. This demonstrates the need of more general defenses, which do not go too deep into details. 

Passwords for Casper41 and Casper42 are, respectively, \\\textit{oSZniFOtdo10zwIgVNvj3usRYtYmJk7g} and \\\textit{BBN7saHY0ULw0VWTdgSKtx2z3RodGEnA}.